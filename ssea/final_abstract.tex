\documentclass[onecolumn]{article}
\usepackage{authblk} % Package for author affiliations
\usepackage{lipsum}  % Remove this in your actual document
\usepackage{hyperref}

\title{Your Conference Paper Title Here}

% Authors with affiliations
\author[1]{Joan Alcaide-Núñez}
\author[2]{Lluc Soler Manich}
\author[3]{Martina Solano Soto}

% Affiliation definitions
\affil[1]{Fakultät für Physik, Ludwig-Maximilians-Universität München, Gescwhister-Scholl-Platz 1, 80539 München, Germany}
\affil[2]{Universitat Politècnica de Catalunya, Spain}
\affil[3]{La Salle Girona, Spain}

\date{\today}

\begin{document}

\maketitle

\begin{abstract}
    We present a concise summary of CAPICR, a cosmic ray detector instrument, and CAPIGX, a student developed, high energy astrophysics mission concept for the 2030s. Where CAPICR and CAPIGX represnt the short (1-3 years) and long (15-20 years) term projects of the CAPIBARA Collaboration.

    Cosmic Rays (CRs) are charged particles formed in outer space with extremely high energies due to relativistic effects composed mainly by protons (78\%) and Helium nuclei (10\%). While CRs come from some of the most extreme environments of the Universe (or from our Sun), Earth's atmosphere prevents the highest energetics CRs from reaching the surface. In order to overcome this limitation, we aim to develop a cosmic ray detector (the CAPICR instrument), which will flight to Low Earth Orbit (LEO) as a payload of the OAB FARADAY satellite by OAB Space.

    The objective is to study the flux and energy spectrum of primary CRs with a system consisting of: a plastic scintillator, a silicon photomultiplier (SiPM), and an integrated circuit. When high energy particles passes through the scintillator, it emits photons collected by SiPM, which generates voltage pulses detected by an electronic circuit. Contained in a 3D-printed metallic enclosure for radiation protection, it enhances the photon collection efficiency. CAPICR benefits from PLD Space's SPARK initiative and the partnership with OBA Space, which provide guidance and technical support.


    CAPICG is the long-term vision of the CAPIBARA Collaboration, aiming to provide reliable, accurate, and complete observations of the high energy transient sky (short-lived phenomena in the X-ray and γ-ray) and provide a platform for multi-messenger sources follow-ups.

    Consisting of a constellation of 3 small CubeSats (presumably 3-4U) CAPIGX will be able to use triangulation and, eventually, intensity interferometry for sub-arcminute source localisation, allowing very precise ground-based follow-ups (IR, optical, radio, UHE) and host galaxy identification. Each CubeSat would have a 1-150 keV Cadmium Zinc Telluride (CZT) detector and a 0.15−2 MeV Cesium Iodide (CsL(TI)) with SiPM detector, covering soft Xray to γ-ray spectrum aside from the necessary equipment.

    This mission was developed because the 2030s will be a "golden" decade for gravitational wave (GW) interferometry, expecting Einstein Telescope, Cosmic Explorer, and LISA to come onlien not before 2035. However, despite the GW detection capabilities, high-energy observatories capable of doing these kind of multi-messenger observations are being planned, aside from THESEUS and NewAthena. CAPIGX will aid these flagship mission by providing more detection points for source localisation, sky coverage, and keeping up with the follow-up demand (simulations predict detections of $10^4$ sources/year by ET).

    The CAPIBARA Collaboration (https://capibara3.github.io) is a group of high school and university students with an endeavour to explore the high energy cosmos. Beyond the scientific goals, our mission is to serve as platform for education and promote student involvement by generating opportunities and valuable data. Highlighting the collaboration between academic and private institutions, we want to show the potential of youth. Remarkably, both instrument and mission designs have been conceived and developed entirely by high-school and undergraduate students.

    \textit{Currently, 3201 (excluding space) characters (max. is 2800)}
    \vspace{0.5em}
    \noindent\textbf{Keywords:} cosmic rays, satellite, education, student-led, high energy astrophysics, transient phenomena
\end{abstract}

\end{document}






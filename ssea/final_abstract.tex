\documentclass[onecolumn]{article}
\usepackage{authblk} % Package for author affiliations
\usepackage{lipsum}  % Remove this in your actual document
\usepackage{hyperref}

\title{Your Conference Paper Title Here}

% Authors with affiliations
\author[1]{Joan Alcaide-Núñez}
\author[2]{Lluc Soler Manich}

% Affiliation definitions
\affil[1]{Fakultät für Physik, Ludwig-Maximilians-Universität München, Gescwhister-Scholl-Platz 1, 80539 München, Germany}
\affil[2]{Universitat Politècnica de Catalunya, Spain}

\date{\today}

\begin{document}

\maketitle

\begin{abstract}
    We present a concise summary of CAPICR, a cosmic ray detector instrument, and CAPIGX, a student developed, high energy astrophysics mission concept for the 2030s. Where CAPICR and CAPIGX represnt the short (1-3 years) and long (15-20 years) term projects of the CAPIBARA Collaboration.

    Cosmic Rays (CRs) are charged particles originating from outer space with extremely high energy due to their relativistic velocities. They composed mainly by protons $(\sim 78\%)$ and Helium nuclei $(\sim 10\%)$. While this come from some of the most extreme environments of the Universe, or directly from our Sun (space weather), Earth's atmosphere prevents most high energy cosmic rays from reaching the surface. Therefore, to overcome this limitation, we aimto develop a cosmic ray detector (the CAPICR instrument), which will flight to Low Earth Orbit (LEO) as a payload of the OAB FARADAY satellite by OAB Space.

    The main goal is to study the flux and energy spectrum of primary CRs with a system consisting of a plastic scintillator, a silicon photomultiplier (SiPM), and an integrated circuit. The principle of operation is simple: when a high energy particles passes through the scintillator, it emits photons collected by the SiPM, which generates voltage pulse detected by the electronic circuit. These elements are contained ina 3D-printed metallic enclosure for radiation protection, enhancing at the same time photon collection efficiency. CAPICR benefits from PLD Space's SPARK initiative and the partnership with OBA Space, who provide guidance and technical support.


    Moreover, CAPICG is the long-term vision of the CAPIBARA Collaboration, aiming to provide reliable, accurate, and complete observations of the high energy transient sky (short-lived phenomena in the X-ray and $\gamma$-ray), in particular providing a platform for multi-messenger sources follow-ups.

    Consisting of a constellation of 3 small CubeSats (presumably 3-4U) CAPIGX will be able of using triangulation and eventually intensity interferometry for sub-arcminute source localisation, allowing for very precise ground-based follow-ups (IR, optical, radio, UHE) and host galaxy identification. In addition to the necessary operations equipment (on-board computer, communication systems, energy, etc.), each CubeSat would have a $1-150\ \mathrm{keV}$ Cadmium Zinc Telluride (CZT) detector and a $150\ \mathrm{keV} - 2\ \mathrm{MeV}$ Cesium Iodide (CsL(TI)) with SiPM detector, covering from soft Xray to $\gamma$-ray spectrum.

    The need for this mission emerges from the fact that the 2030s will present a golden era for gravitational wave interferometry, expecting Einstein Telescope, Cosmic Explorer and LISA to come onlien not before 2035. However, despite the GW detection capabilities any high-energy observatories for doing these kind of multi-messenger observations are being planned, aside from THESEUS and NewAthena. CAPIGX will aid these flagship mission by providing more detection points for source localisation, sky coverage, and keeping up with the follow-up demand (simulations predict detections of $\sim 10^4$ sources/year by ET).

    The CAPIBARA Collaboration is a group of high school and university students with the aim to explore the high energy cosmos (https://capibara3.github.io). Beyond the scientific goals, our mission is to serve as platform for education and promote student involvement by generating opportunities and valuable data. Highlighting the collaboration between academic and private institutions at all levels through education, we want to show the potential of youth. Remarkably, both instrument and mission designs have been conceived and developed entirely by high-school and undergraduate students, demonstrating our commitment.

    \textit{Currently, 3201 (excluding space) characters (max. is 2800)}
    \vspace{0.5em}
    \noindent\textbf{Keywords:} cosmic rays, satellite, education, student-led, high energy astrophysics, transient phenomena
\end{abstract}

\end{document}






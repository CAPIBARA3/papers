\documentclass[onecolumn]{article}
\usepackage{authblk} % Package for author affiliations
\usepackage{lipsum}  % Remove this in your actual document
\usepackage{hyperref}

\title{Your Conference Paper Title Here}

% Authors with affiliations
\author[1]{Joan Alcaide-Núñez}
\author[2]{Lluc Soler Manich}
\author[3]{Martina Solano Soto}

% Affiliation definitions
\affil[1]{Fakultät für Physik, Ludwig-Maximilians-Universität München, Gescwhister-Scholl-Platz 1, 80539 München, Germany}
\affil[2]{Universitat Politècnica de Catalunya, Spain}
\affil[3]{La Salle Girona, Spain}

\date{\today}

\begin{document}

\maketitle

\begin{abstract}
    We present a concise summary of CAPICR, a cosmic ray detector instrument, and CAPIGX, a student developed, high energy astrophysics mission concept for the 2030s. Where CAPICR and CAPIGX represent the short (1-3 years) and long (15-20 years) term projects of the CAPIBARA Collaboration.


   Cosmic Rays (CRs) are charged, highly energetic particles with relativistic speeds coming from outer space or the Sun. Protonsmaking 78\% and Helium 10\% of primary CRs, Earth’s atmosphere prevents them from reaching the surface. To overcome this limitation, we aim to develop a CR detector (CAPICR) which will flight to LEO as a payload on OAB FARADAY by OBA Space.


   The goal is to study the flux and energy spectrum of primary CRs with a system consisting of a plastic scintillator, a silicon photomultiplier (SiPM), and an integrated circuit. When CR particles pass through the scintillator, it emits photons collected by the SiPM, generating a voltage pulse detected by an electronic circuit. Contained in a 3D-printed metallic enclosure for radiation protection, it enhances the photon collection efficiency. CAPICR benefits from PLD Space's SPARK program and the partnership with OBA Space, which provide guidance and technical support.


   CAPICG is the long-term vision of CAPIBARA, aiming to provide accurate and reliable observations of the high energy transient sky (short-lived phenomena) and provide a platform for multi-messenger event follow-ups.


   Consisting of a constellation of 3 small CubeSats (presumably 3-4U) CAPIGX will be able of triangulation and possibly intensity interferometry for subarcminute localisation, allowing precise ground follow-up and host galaxy identification. Each CubeSat would have a 1-150 keV and a 0.15−2 MeV detector, covering X-ray to γ-ray spectrum aside from the necessary equipment.


    This mission fits into the “golden decade” for multi-messenger astronomy with multiple GW observatories going online in 2035 (Einstein Telescope, Cosmic Explorer, LISA). While ESA plans THESEUS and NewAthena, CAPIGX will aid these flagship missions by providing sky coverage, and keeping up with transient follow-ups (simulations predict 10000 binary neutron star detections by ET per year).


   The CAPIBARA Collaboration (https://capibara3.github.io) is a group of high school and university students with an endeavour to explore the high energy cosmos. Beyond the scientific goals, our mission is to serve as platform for education and promote student involvement by generating opportunities and valuable data. Highlighting the collaboration between academic and private institutions, we want to show the potential of youth. Remarkably, both instrument and mission designs have been conceived and developed entirely by high-school and undergraduate students.


    \textit{Currently, 3201 (excluding space) characters (max. is 2800)}
    \vspace{0.5em}
    \noindent\textbf{Keywords:} cosmic rays, satellite, education, student-led, high energy astrophysics, transient phenomena
\end{abstract}

\end{document}






\documentclass[onecolumn]{article}
\usepackage{authblk} % Package for author affiliations
\usepackage{lipsum}  % Remove this in your actual document
\usepackage{hyperref}

\title{Your Conference Paper Title Here}

% Authors with affiliations
\author[1]{Joan Alcaide-Núñez}
\author[2]{Second Author}

% Affiliation definitions
\affil[1]{Fakultät für Physik, Ludwig-Maximilians-Universität München, Gescwhister-Scholl-Platz 1, 80539 München, Germany}
\affil[2]{Research Laboratory, Institution Two}

\date{\today}

\begin{document}

\maketitle

\begin{abstract}
    We present a concise summary of *CAPIGX*, a student developed, mid-2030s, high energy astrophysics mission concept.

    The goal of the CAPIGX mission is to provide reliable, accurate and complete observations of high energy astrophysical transients (short-lived phenomena visible in the X-ray and $\gamma$-ray), in particular providing a platform for the follow-up of multi-messenger astronomical sources.

    Consisting of a constellation of 3 small CubeSats (presumably 2-4U) CAPIGX will be able of using triangulation and eventually intensity interferometry to perform high energy source localisations with extreme precision, allowing for better studies of high energy phenomena as well as targeted follow-ups in other wavelengths (IR, optical, radio, UHE).

    In addition to the necessary operations equipment (on-board computer, communication systems, energy, etc.), each CubeSat would have a $1-150 \mathrm{keV}$ ... detector (soft X-ray band) and a $0.15-2\ \mathrm{MeV}$ ... (hard X-ray to $\gamma$-ray) detectors. Thus the mission will cover from soft X-ray to $\gamma$-ray part of the spectrum.

    The need for this mission emerges from the fact that the 2030 will present a golden era for gravitational wave interferometry, expecting Einstein Telescope and Cosmic Explorer (LISA too) to come online not before 2035. However, despite the GW detection capabilities any high-energy observatories for doing these kind of multi-messenger observations are being planned, aside from flagship mission like THESEUS and NewAthena which will not be able to fulfill the observational demand (simulations predict ($10^4$ sources/year for ET alone).

    CAPIGX is a mission proposed by the CAPIBARA Collaboration, a group of high school and university students with the aim to explore the high energy cosmos. More information on our website: \href{https://capibara3.github.io}{https://capibara3.github.io}

    Recently, we have been outlining the details of our mission and having throughout discussions within the collaboration. We are also currently developing simulations and scientific outcome expectations/requirements, as well as writing a Mission Concept Documentation (MCD) paper.

    \textit{Currently, 2157 characters (max. is 2800)}
    \vspace{0.5em}
    \noindent\textbf{Keywords:} keyword1, keyword2, keyword3, keyword4
\end{abstract}

\end{document}

\documentclass{article}
\usepackage{authblk} % Package for author affiliations
\usepackage{lipsum}  % Remove this in your actual document

\title{CAPICR: A Student-Led Satellite Mission for Cosmic Ray Detection}

% Authors with affiliations
\author[1]{qui presentarà? a munich}
\author[1]{Lluc Soler}
\author[1]{Third Author}

% Affiliation definitions
\affil[1]{Collaboration for the Analysis of Photonic and Ionic Bursts and Radiation (CAPIBARA Collaboration)}


\date{}

\begin{document}

\maketitle

\begin{abstract}
Cosmic Rays (CRs), or cosmic radiation, are charged particles originating from outer space with extremely high energy due to their relativistic velocities. These particles are mainly protons (\sim 87\%) and Helium nuclei (\sim 10\%), and they provide key insights in the astrophysics field. However, Earth’s atmosphere prevents most high-energy photons (X-rays and \gamma-rays) and primary cosmic rays from reaching the surface. Therefore, to overcome these limitations, we aim to develop a compact cosmic ray detector (the CAPICR instrument), which will be sent to a Low Earth Orbit (LEO).

The mission’s main goal is to study the flux and energy spectrum of primary cosmic rays. To achieve this, CAPICR’s detection system will consist of a plastic scintillator, a silicon photomultiplier (SiPM), and an integrated circuit. The principle of operation is simple: when a high-energy particle passes through the scintillator, it will emit photons, which will be collected by the SiPM, and detected as a voltage pulse in the electronic circuit. To protect all these elements, they will be contained in a 3D-printed metallic enclosure, which, thanks to its intrinsic properties, enhances photon collection efficiency by redirecting the photons toward the SiPM. This mission is part of the PLD Space Spark initiative, which aims to provide better and more affordable access to space by promoting the development of student-led and educational research projects. Therefore, CAPICR benefits form the support of key partners, such as Orbital Boost Aerospace (OBA). Within the framework of the collaboration, our detector will be integrated into the OBA’s FARADAY CubeSat satellite, which will provide a strong infrastructure for the instrument. 

Beyond the scientific goals, this mission also aims to serve as a platform for education and student involvement in space science. By generating open scientific data, we aim to promote student hands-on student contribution in astrophysics and engineering research, as well as highlight the importance of collaboration between academic and private institutions, young researchers, and instructors. Remarkably, both the instrument and the mission design have been conceived and developed entirely by high-school students and undergraduates, demonstrating our commitment to education through experience. CAPICR, represents not only a scientific project, but also an attempt to inspire the next generation of researchers. 
    
    \vspace{0.5em}
    \noindent\textbf{Keywords:} astrophysics, cosmic rays, satellite, education
\end{abstract}

\end{document}

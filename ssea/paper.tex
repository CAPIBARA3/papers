\documentclass[11pt]{article}

% --- Packages ---
\usepackage{amsmath,amssymb}    % math symbols
\usepackage{graphicx}           % figures
\usepackage{array}              % tables
\usepackage{booktabs}           % better tables
\usepackage{geometry}           % page layout
\usepackage{natbib}             % bibliography
\usepackage{hyperref}           % clickable links
\hypersetup{
    colorlinks=true,
    linkcolor=cyan,
    filecolor=magenta,      
    urlcolor=blue,
    pdftitle={Overleaf Example},
    pdfpagemode=FullScreen,
}
\geometry{a4paper, margin=1in}

\title{SSEA Paper CAPIBARA Collaboration\\
\textcolor{red}{depending on how much space figures/references take, we should aim for ~2500 words following the SSEA restrictions. WHen writing account for max 300 words per section, and max-max 400 words for sec 3 and 4 each}}

\author{
    Joan Alcaide-Núñez\thanks{Ludwig-Maximilians-Universität München (LMU), Munich, Germany} \and
    Lluc Soler Manich\thanks{Universitat Politècnica de Catalunya (UPC), Barcelona, Spain} \and
    Carles Fonseca Maurí\thanks{University of California - Los Angeles, Los Angeles, USA} \and
    Alessandro Sabia\thanks{affiliation} \and
    Martina Solano Soto\thanks{affiliation} \and
    Rosa Bermejo Sabugo\thanks{Universidad de Salamanca (USal), Salamanca, Spain} \and
    Hongda Zheng Xu\thanks{affiliation} \and
    
    % check before adding
    % Martí Delgado Farriol\thanks{affiliation} \and
    % Anna Abadal Garrido\thanks{affiliation} \and
    % Marc 
}

\date{\today}

\begin{document}

\maketitle

\begin{abstract}

    This is our abstract.

\end{abstract}

\textbf{Keywords:} keyword1, keyword2, keyword3

\section{Introduction}

The landscape of modern space exploration is shifting toward more inclusive, decentralised models, where student-led missions serve as vital engines for both technical innovation and the cultivation of the next generation of space professionals. The rise of 

Why student-led missions matter

What CAPIBARA is (scope, student-led philosophy, advisors' and partnerships' role)

High-energy astrophysics as principal goal

multiple missions to pursue that goal

The landscape of modern space exploration is shifting toward more inclusive, decentralised models, where student-led missions play relevant roles in both technical innovation and the cultivation of the next generation of space professionals. The rise of CubeSats in the decade as enablers of space-access for student-led missions is allowing for students to get early experience and expertise in space engineering, fostering new talents. For instance, the 14 CubeSats have been launched by students from Europe via the European Space Agency’s (ESA) FlyYourSatellite! Programme (FYS).

The Collaboration for the Analysis of Photonic and Ionic Bursts And RAdiation (CAPIBARA Collaboration) is a group of students, consisting of high school and undergraduate students for now, with the goal to explore the high energy cosmos in every aspect. The educational aim of CAPIBARA is to foster early interests and talents creating a community to both test one’s interests and skills and contribute to real-world science.

To pursue our goals we separated them into two aspects: ionic and photonic detections, and the following that two missions, each of them with increasing complexity and ambition.


\section{The CAPIBARA Collaboration Framework}

advisors as mentors not mission leader

full student-led but distributed and organised model
mission-based participation

allows students to engage with real-world science/engineering and to test their skills and interests

progressive increase in mission complexity, early missions validate, later missions expand ambition and collaboration

The key features of the CAPIBARA Collaboration are (+or- summary):
Open science/access basis
Open to every students (from high school to graduate students)
But only students, advisors and partners are mentors to us not leaders
Learn by doing and learn by sharing
Dual goals, dual mission


\section{Cosmic Ray Detector}

objectives and scientific rationale

detector overview (principle, challenges, expected outcome)

status

\section{C...O...S...M...O...S...}

\textbf{Purpose is to define our vision for Duo and Net and invite SSEA community members (across academia, industry and research) to become mentors}

Building upon the heritage of past missions, the COSMOS program represents the long-term strategic vision of CAPIBARA, aiming to bridge the gap in high-energy data availability throughout the 2030s through a phased multi-platform mission architecture.

scientific rationale and phased evolution

leverage heritage missions and integrate into a gap for the 2030s

grow with new membesr and advisors


In order to fulfil CAPIBARA’s goal to observe the high-energy universe in gamma-rays and X-rays (high energy electromagnetic spectrum) we have envisioned the COSMOS program as our long-term aim into the next decade. COSMOS, with its phased approach, aims to contribute to high-energy transient events observations and localisation of those events. These localisations would provide vital information for follow-ups in other electromagnetic (EM) wavelengths (X-ray, IR, optical, radio) as well as other ‘messengers’, i.e. gravitational waves signals (GW) and neutrino detections by other observatories both space- and ground-based.

We are currently exploring this project as a constellation of small CubeSat observatories. When multiple satellites observe the same event, their positions and measurements can be used to further localise the event’s coordinates. This principle was already proven by the interplanetary network (IPN) and the general coordinates network (GCN) and more recently being also explored by the HERMES pathfinder mission.

CAPIBARA aims to leverage the high-energy detector heritage by past CubeSat missions such as EIRSAT-1, HERMES and BurstCube. Our intention is to implement flight-proven detector technology to 



\section{Discussion}

contrast both missions

how CRD validates the framework

CAPIBARA as a platform for student growth and development in all roles/fields

why cosmos also represents CAPIBARA

emphasise collaboration sustainability and expansion for 2030s

CRD proves we can build, COSMOS invites you to build with us


\section{Conclusion}\label{sec:conclusion}

Summary and final call, why this matters

This is our Conclusion.

\section*{Acknowledgments}

We are grateful to the members of the CAPIBARA Collaboration for useful discussions and feedback. The authors are happy to welcome feedback and advice, please contact the corresponding author or visit our website: \href{https://capibara3.github.io/contact}{capibara3.github.io/contact}.

% We thank the developers of the open-source tools used in this work. 

% --- References ---
% \bibliographystyle{plain}
\bibliographystyle{unsrt}
\bibliography{references}


\end{document}

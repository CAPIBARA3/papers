\documentclass[11pt]{article}

% --- Packages ---
\usepackage{amsmath,amssymb}    % math symbols
\usepackage{graphicx}           % figures
\usepackage{array}              % tables
\usepackage{booktabs}           % better tables
\usepackage{geometry}           % page layout
\usepackage{natbib}             % bibliography
\usepackage{hyperref}           % clickable links
\hypersetup{
    colorlinks=true,
    linkcolor=cyan,
    filecolor=magenta,      
    urlcolor=blue,
    pdftitle={Overleaf Example},
    pdfpagemode=FullScreen,
}
\geometry{a4paper, margin=1in}

\title{SSEA Paper CAPIBARA Collaboration\\
\textcolor{red}{depending on how much space figures/references take, we should aim for ~2500 words following the SSEA restrictions. WHen writing account for max 300 words per section, and max-max 400 words for sec 3 and 4 each}}

\author{
    Joan Alcaide-Núñez\thanks{Ludwig-Maximilians-Universität München (LMU), Munich, Germany} \and
    Lluc Soler Manich\thanks{Universitat Politècnica de Catalunya (UPC), Barcelona, Spain} \and
    Carles Fonseca Maurí\thanks{University of California - Los Angeles, Los Angeles, USA} \and
    Alessandro Sabia\thanks{affiliation} \and
    Martina Solano Soto\thanks{affiliation} \and
    Rosa Bermejo Sabugo\thanks{Universidad de Salamanca (USal), Salamanca, Spain} \and
    Hongda Zheng Xu\thanks{affiliation} \and
    
    % check before adding
    % Martí Delgado Farriol\thanks{affiliation} \and
    % Anna Abadal Garrido\thanks{affiliation} \and
    % Marc 
}

\date{\today}

\begin{document}

\maketitle

\begin{abstract}

    This is our abstract.

\end{abstract}

\textbf{Keywords:} keyword1, keyword2, keyword3

\section{Introduction}

The landscape of modern space exploration is shifting toward more inclusive, decentralised models, where student-led missions serve as vital engines for both technical innovation and the cultivation of the next generation of space professionals. The rise of 

Why student-led missions matter

What CAPIBARA is (scope, student-led philosophy, advisors' and partnerships' role)

High-energy astrophysics as principal goal

multiple missions to pursue that goal

\section{The CAPIBARA Collaboration Framework}

advisors as mentors not mission leader

full student-led but distributed and organised model
mission-based participation

allows students to engage with real-world science/engineering and to test their skills and interests

progressive increase in mission complexity, early missions validate, later missions expand ambition and collaboration

\section{Cosmic Ray Detector}

objectives and scientific rationale

detector overview (principle, challenges, expected outcome)

status

\section{C...O...S...M...O...S...}

\textbf{Purpose is to define our vision for Duo and Net and invite SSEA community members (across academia, industry and research) to become mentors}

Building upon the heritage of past missions, the COSMOS program represents the long-term strategic vision of CAPIBARA, aiming to bridge the gap in high-energy data availability throughout the 2030s through a phased multi-platform mission architecture.

scientific rationale and phased evolution

leverage heritage missions and integrate into a gap for the 2030s

grow with new membesr and advisors


\section{Discussion}

contrast both missions

how CRD validates the framework

CAPIBARA as a platform for student growth and development in all roles/fields

why cosmos also represents CAPIBARA

emphasise collaboration sustainability and expansion for 2030s

CRD proves we can build, COSMOS invites you to build with us


\section{Conclusion}\label{sec:conclusion}

Summary and final call, why this matters

This is our Conclusion.

\section*{Acknowledgments}

We are grateful to the members of the CAPIBARA Collaboration for useful discussions and feedback. The authors are happy to welcome feedback and advice, please contact the corresponding author or visit our website: \href{https://capibara3.github.io/contact}{capibara3.github.io/contact}.

% We thank the developers of the open-source tools used in this work. 

% --- References ---
% \bibliographystyle{plain}
\bibliographystyle{unsrt}
\bibliography{references}


\end{document}
